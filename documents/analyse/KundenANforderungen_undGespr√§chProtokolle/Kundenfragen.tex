%% LyX 1.6.3 created this file.  For more info, see http://www.lyx.org/.
%% Do not edit unless you really know what you are doing.
\documentclass[ngerman, a4paper]{article}
\usepackage[T1]{fontenc}
\usepackage[latin9]{inputenc}
\usepackage{babel}
\usepackage{a4wide}

\begin{document}

\title{Fragen an die Kunden:}

\maketitle

\section{Ist Analyse}
\begin{itemize}
\item Welche Konzepte/Ans�tze werden momentan genutzt und wof�r?
\item Mit welchen Konzepten wird bis jetzt auf wissenschaftliche Daten �ber
BPEL-Prozesse zugegriffen, wenn �berhaupt?
\item Welche Datenquellen werden bereits benutzt?
\end{itemize}

\section{Angebot}
\begin{itemize}
\item Welche speziellen Inhalte soll das Angebot enthalten?
\item Strafzahlungen pro Tag Versp�tung. (Notenabzug)
\item Welche Erreichbarkeit des Service wird gew�nscht?
\end{itemize}

\section{Generische BPEL Aktivit�ten }
\begin{itemize}
\item Replacement, Delete in Klammern bedeutet optional, oder wenn m�glich?
(siehe Folie Nr.7 in {[}\ref{the:Kick-Off-Folien}{]})
\item Anlegen von DB-Tabellen wird gew�nscht, auch �nderung und L�schen
von Tabellen? (siehe Folie Nr.7 in {[}\ref{the:Kick-Off-Folien}{]})
\item BPEL-Aktivit�ten von uns zu erstellen oder bereits irgendwie vorhanden
(m�ssen nur noch angepasst werden)?
\end{itemize}

\section{Anbindung versch. Datenquellen }
\begin{itemize}
\item Welche? Theoretisch alle?
\item Welche Dateiformate m�ssen unterst�tzt werden (lesen/schreiben)?
\item Welche Anfragesprachen f�r den Datenbankzugriff m�ssen unterst�tzt
werden?
\item In welchem Umfang sollen Transaktionen unterst�tzt werden und welche
Funktionen/Konzepte (2-Phase-Commit, ChangeHistory, ...) sollen integriert
sein? Transaktionen �ber mehrere Datenquellen?
\item Wie sollen Anforderungen an eine Datenquelle durch den User modelliert
werden k�nnen? Anotationen?
\item Sollen Daten aus Datenbanken in einem Prozess auch in lokale Dateien
exportiert werden k�nnen und wenn ja in welche Formate?
\item sollen alle Operationen atomar sein, wenn nicht: welche Operationen
sollen als Transaktion durchgef�hrt werden (bzw. wo ist Atomizit�t
notwendig, wo nicht?)
\item Modellierung der Anforderungen an eine Datenquelle

\begin{itemize}
\item bedeutet das ein Nutzer kann eingeben was die Datenquelle k�nnen muss
und das Programm w�hlt eine passende aus?
\item wie soll die Modellierung der Anforderungen genau gesehen, was muss
enthalten sein?
\item Woran soll sich die Strategie orientieren, bzw. Sich haupts�chlich
orientieren?
\end{itemize}
\item Was ist unter {}``Strategien'' zu verstehen?
\item welche Kriterien sollen die Strategien enthalten (Auf welche Art sollen
die Datenbanken/Datenquellen beschrieben werden, was soll alles enthalten
sein) (siehe Folie Nr.8 in {[}\ref{the:Kick-Off-Folien}{]} )
\item wie sollen die Annotations ausgewertet werden, bzw. wie sollen die
Informationen genutzt werden?
\item Gibt es Datenquellen f�r Testzwecke?
\end{itemize}

\section{Autorisierung der Zugriffe }
\begin{itemize}
\item Sollen Authorisierung und Authentifizierung nur f�r Datenquellen bereitgestellt
werden?
\item Welches Autorisierungsverfahren ist gew�nscht? Gibt es bereits Zugriffsregeln?
\item Authentifizierung wird gar nicht als Anforderung genannt.
\item Authentifizierung der Zugriffe auf eine bestimmte Art, oder auf verschiedene
Arten?
\item Gibt es bereits ein Authentifizierungsverfahren bei dem Zugriff auf
die wissenschaftlichen Datenbanken?
\item Wo und in welchem Umfang soll die Autorisierung und die Authentifizierung
stattfinden?
\item Beschr�nken sich Autorisierung und Authentifizierung nur auf die Ausf�hrung
oder auch auf die Modellierung/Implementierung von Prozessen (z.B.
bei Implementierung von abstrakten Prozessen)?
\item Wie und wo sollen Autorisierungs- und Authentifizierungsparameter
f�r einen Prozess spezifiziert werden?
\item Autorisierung bei allen Datenquellen notwendig? wenn ja auf die selbe
Art? wenn nein bei welchen ist es notwendig?
\end{itemize}

\section{Monitoring der Prozessausf�hrung }
\begin{itemize}
\item Extra Datenbank f�r das Monitoring kann eine feste frei w�hlbare sein?
Oder Unterst�tzung verschiedener Datenbanken?
\item Wo soll das Monitoring ansetzen, bei BPEL Aktivit�ten oder bei Datenbankzugriffen?
Wie ausf�hrlich soll das erfasst werden, welche Informationen werden
ben�tig?
\item wie lang sollen Daten gespeichert werden (History = 2 wochen, 1 Monat
... ?)
\item Monitoring immer aktiv, oder nur auf Userwunsch
\end{itemize}

\section{Sonstige Fragen }
\begin{itemize}
\item Wie soll die Demo genau aussehen? Was soll gezeigt werden?
\item Welche Schritte werden bei der Prozessmodellierung ausgef�hrt, die
die Funktionalit�t des Rahmenwerks nutzen?
\item In welcher Form soll das Rahmenwerk am Ende zur Verf�gung stehen?
\item Welche speziellen Dokumente m�ssen am Ende des Projekts angefertigt
worden sein? (Erweiterte BPEL-Spec, ...)
\item Stehen die Kunden w�hrend des Projekts f�r Reviews oder �hnliches
zu Verf�gung?
\item Bedeutung/Definition von Erweiterbarkeit des Frameworks? (siehe {[}\ref{the:Ausschreibung}{]})
\item Welche Teile/Aspekte sollen erweiterbar sein? (siehe {[}\ref{the:Ausschreibung}{]})
\item Welche Erfahrungen haben die Nutzer?
\item Zielgruppe?
\item �berpr�fung der Benutzereingaben?
\item Wird eine bestimmte Middleware bevorzugt?
\end{itemize}

\section{Nichtfunktionale Anforderungen }
\begin{itemize}
\item Fehlertoleranz?
\item Ausfallsicherheit?
\item Erweiterbarkeit?
\item Anforderung an Benutzbarkeit(usability)\end{itemize}

\begin{thebibliography}{2}
\bibitem{key-1}\label{the:Ausschreibung}Projektantrag\_StuProA\_ODE.pdf

\bibitem{key-2}\label{the:Kick-Off-Folien}SIMPL\_KickOff.pdf
\end{thebibliography}

\end{document}
